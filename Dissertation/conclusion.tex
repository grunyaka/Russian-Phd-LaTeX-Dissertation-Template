\chapter*{Заключение}                       % Заголовок
\addcontentsline{toc}{chapter}{Заключение}  % Добавляем его в оглавление

%% Согласно ГОСТ Р 7.0.11-2011:
%% 5.3.3 В заключении диссертации излагают итоги выполненного исследования, рекомендации, перспективы дальнейшей разработки темы.
%% 9.2.3 В заключении автореферата диссертации излагают итоги данного исследования, рекомендации и перспективы дальнейшей разработки темы.
%% Поэтому имеет смысл сделать эту часть общей и загрузить из одного файла в автореферат и в диссертацию:

Основные результаты работы заключаются в следующем.
%% Согласно ГОСТ Р 7.0.11-2011:
%% 5.3.3 В заключении диссертации излагают итоги выполненного исследования, рекомендации, перспективы дальнейшей разработки темы.
%% 9.2.3 В заключении автореферата диссертации излагают итоги данного исследования, рекомендации и перспективы дальнейшей разработки темы.
\begin{enumerate}
 \item Для выполнения поставленных задач были проведены наблюдения на Нормальном астрографе и телескопе <<Сатурн>>;
 \item Для обработки полученных наблюдений был исследован и адаптирован метод shapelet-разложения;
   \item Для получения прочих материалов исследования, численных расчетов и построения промежуточных моделей было создано специализированное программное обеспечения на C++ и Python;
 \item На основе анализа собственных движений 1308 быстрых звезд было выявлен 121 кандидат в $\Delta\mu$-двойные, по данному исследованию опубликована работа \cite{2015AstL...41..833K};
 \item Были проведены дополнительные спекл-интерферометрические исследования 7 программных звез, для пяти из которых были подтверждены статусы двойных (присутствие 2х из них в каталоге WDS может служить фактором верификации исследования), по выявлению одной из звезд опубликована работа \cite{2016AstL...42..686K} и уже получены наблюдения, позволяющие построить её предварительную орбиту;
   \item В результате анализа форм изображений 702 звезд (отмеченных в Gaia DR2 флагом \glqq duplicate source\grqq ) было выявлено ещё 138 кандидатов в двойные, по данному исследованию также написана статья \cite{2018AstL...44..103K}.
\end{enumerate}

В качестве основного итога проведенного исследования следует отметить, что был разработан и реализован достаточно эффективный метод выявления двойных систем среди близких карликов, включающий детальное исследование собственных движений звезд и анализ формы изображений, а также верификацию отобранных объектов посредством спекл-наблюдений.

После выхода Gaia DR2 стало понятно, что проект испытывает некоторые трудности с обработкой быстрых звезд, остро встал вопрос кросс-идентификации объектов. Это повлекло неполноту релиза в части близких карликов. Также авторы отметили, что вызывают сложности с разрешением звездные системы теснее $\rho\,<\,2''$. В документации Gaia DR2 особо отмечается, что ряд объектов, обозначенных флагом <<duplicate source>>, могут быть как кратными звездными системами, так и дубликатами одной и той же звезды, полученной из-за большой величины её собственного движения. Эти объекты предлагается исследовать отдельно с активным привлечением в том числе и наземных наблюдений. Новые высокоточные собственные движения звезд были сравнены с собственными движениями выделенных ранее звезд-кандидатов в $\Delta\mu$-двойные. Тот факт, что не все звезды пулковской программы оказались во втором релизе Gaia, подтверждает проблему кросс-идентификации в космической миссии и выявляет неполноту Gaia DR2 в плане близких карликов. Однако высокие значения параметра F, рассчитанного с участием собственных движений Gaia, подтверждают состоятельность проведенного нами ранее исследования.

В будущем предполагается провести полноценное исследование по анализу собственных движений быстрых звезд с активным привлечением данных Gaia. Помимо спекл-наблюдений для верификации кандидатов в двойные звезды и построения их взаимных орбит предполагается привлечь технологию <<Lucky imaging>>(метод удачных экспозиций).

\newpage
\begin{center}
\textbf{Благодарности}
\end{center}


Куликова А. благодарит за неоценимую помощь в работе и написании диссертации своего научного руководителя Максима Юрьевича Ховричева. Кроме того:
\begin{itemize}
  \item Наблюдателей Нормального астрографа и телескопа <<Сатурн>> Пулковской обсерватории, соавторов статьи \cite{2018AstL...44..103K}:
  \begin{itemize}
    \item Апетян А.~А.;
    \item Рощину Е.~А.;
    \item Измайлова И.~С.;
    \item Бикулову Д.~А.;
    \item Ершову А.~П.;
    \item Баляева И.~А.;
    \item Петюра В.~В.;
    \item Шумилова А.~А.;
    \item Оськину К.~И.;
    \item Максимову Л.~А.;
  \end{itemize}
  \item Коллег и соавторов статьи \cite{2016AstL...42..686K} из ГАО РАН, САО РАН И ГАИШ МГУ:
  \begin{itemize}
    \item Сокова Е.~Н.;
    \item Дьяченко В.~В.;
    \item Растегаева Д.~А.;
    \item Бескакотова А.~С.;
    \item Балегу Ю.~Ю.;
    \item Сафонова Б.~С.;
    \item Додина А.~В.;
    \item Вознякову О.~В.;
  \end{itemize}
\end{itemize}
А также создателей CDS\footnote{\textit{http://cds.u-strasbg.fr}} за возможность пользования базой и авторов оригинального русскоязычного шаблона <<Russian-Phd-LaTeX-Dissertation-Template>>\footnote{\textit{https://github.com/AndreyAkinshin/Russian-Phd-LaTeX-Dissertation-Template}}, который был использован при написании диссертации.
