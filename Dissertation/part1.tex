\chapter{Астрометрический подход к поиску двойных систем} \label{ch:ch1}
\section{Исторический опыт астрометрического исследования быстрых звезд} \label{sec:ch1/sec1}
Развитие методов фотографической астрометрии на рубеже XIX--XX веков позволило начать массовое определение собственных движений звезд, не входящих в каталоги, построенные на основе меридианных наблюдений. Следствием этого стало открытие в 1916 году Эдвардом Барнардом звезды, с самым большим собственным движением \cite{1916AJ.....29..181B}. Сравнительно быстрое перемещение звезды Барнарда на фоне соседей ($\mu$~=~10.358~$''$/yr) закрепило за ней название <<летящей>>. Звезда Барнарда не является ближайшей к Солнцу, однако ожидаемо входит в наиболее тесное с нами звёздное соседство. Расстояние до <<летящей>> звезды чуть более 1.8~пк, и она является четвёртой известной звездой по мере удаления от Солнца, уступая в близости только системе звёзд Альфа Центравра. Однако, помимо выдающегося собственного движения, она имеет и значительную величину лучевой скорости, при этом приближаясь к Солнцу (\(\textup{М}_r\)~>~110 км/сек), и по оценкам может обогнать ближайшую к нам систему звёзд примерно к 11800 году. Стоит также отметить, что звезда Барнарда является красным карликом "--- представителем одной из наиболее многочисленных групп ближайшего околосолнечного населения Галактики. 

Значительные величины собственных движений звезд дают простор для приложения астрометрических методов поиска двойных и кратных объектов. Здесь на самом деле речь идет о весьма разнообразных подходах. Наиболее старый заключается в попытке обнаружить орбитальное движение для хорошо разрешаемых звездных пар. Эта задача вышла на передний план развития наблюдательной астрономии в конце XVIII века. В 1803 году в мемуарах Гершеля было впервые надежно показано наличие орбитального движения. Этот метод, в основном, касается широких пар с относительно большими периодами обращения (сотни и тысячи лет). Если иметь ввиду солнечную окрестность, то в ее пределах обнаружены десятки двойных систем именно таким способом. От 61-ой Лебедя до современных наблюдений (например, проект RECONS\footnote{\textit{http://www.recons.org/}}). 

Массовые и сравнительно точные обзоры собственных движений звезд сразу позволили выявить пары, компоненты которых характеризуются <<общим>> собственным движением. Действительно, для широких пар  скорость движения относительно Солнца заметно больше скорости их взаимного орбитального движения. Поэтому малые различия собственного движения между компонентами оптически двойной с большой вероятностью означают физичность пары. Если к этому добавляется еще и приблизительное равенство лучевых скоростей, тогда вопрос об обнаружении двойной системы можно считать практически решенным.

Реализация миссии Hipparcos больше четверти века назад, и появление первых релизов миссии Gaia повысили интенсивность подобных поисков (например, \cite{2018JDSO...14..367K}). Выявлено множество широких пар двойных звезд на основе совместного анализа всей астрометрической информации: параллаксов, собственных движений и лучевых скоростей (например, \cite{2019A&A...623A..72K}).

Отдельного рассмотрения заслуживает метод обнаружения так называемых астрометрических двойных звезд (или звезд с невидимыми спутниками). Речь идет о том, что для неразрешаемой при обычных наблюдениях двойной звезды может иметь место значимое различие положений фотоцентра и центра масс. В этом случае наблюдаемое движение звезды становится волнообразным. Наиболее яркий пример "--- детектирование Фридрихом Бесселем невидимых спутников Сириуса и Проциона. В его работе были проанализированы движения ярчайших звезд неба на основе данных разных обсерваторий за несколько десятилетий. К 1844 году после многолетних наблюдений Проциона и Сириуса Бессель опубликовал результаты вычислений, которые говорили о том, что движения этих звезд имеют заметные периодические отклонения от своих средних долгопериодических трендов, то есть не отличаются прямолинейностью. Наблюдаемые фотоцентры описывали волнообразные линии, что говорило о наличии невидимых спутников у обеих звезд. Позднее эти выводы были подтверждены двумя американскими астрономами. В 1862 году Алван Кларк и в 1896 году Джон Шеберле сумели пронаблюдать ранее невидимые компоненты систем Сириуса (спектральный класс первой компоненты "--- A1) и Проциона (класс F) соответственно. Эти спутники, характеризуются низкой светимостью и оказались белыми карликами. Данный подход нашел свое развитие и в эпоху космической астрометрии и оказался приемлемым для сравнительно ярких звезд с хорошей историей астрометрических наблюдений. Примером такого исследования является работа \cite{2002A&A...391..647G}.

\section{Состоятельность задачи поиска $\Delta\mu$-двойных среди близких карликов}\label{sec:ch1/sectN2}
\subsection{Некоторые свойства видимых орбит фотоцентров маломассивных двойных систем}\label{subsec:ch1/sect2/sub1}
Существует целый ряд причин, которые могут приводить к отнесению звезды к категории кандидатов в $\Delta\mu$-двойные. Учитывая небольшое количество наблюдений, характерное для близких карликов, это могут быть просто систематические ошибки. Возможен вариант, когда звезда уже входит в состав двойной системы и обе компоненты видны на ПЗС-кадре как отдельные объекты (разделяются), но ранее эти звезды не были включены в списки известных двойных. При этом одна из компонент может и вовсе иметь блеск, уступающий предельному для  анализируемых каталогов или цифровых кадров. В этом случае сравнение собственных движений по рассмотренной методике может дать значимое отклонение квазимгновенного  собственного движения от квазисреднего. И это будет свидетельством в пользу обнаружения орбитального движения.

Но все же для нас наиболее интересен случай неразрешенности изображений звезд на ПЗС-кадрах. Качественная картина возможности выявления $\Delta\mu$-двойных продемонстрирована в предыдущей главе. Здесь мы попытаемся дать количественные оценки.

\begin{figure}[pt]\label{fig:semiaxis1}
\centering
\includegraphics[width=0.6\textwidth]{semiaxis_vs_mass035_10_25.png}\\
\includegraphics[width=0.6\textwidth]{semiaxis_vs_mass073_10_25.png}
\caption{Зависимость большой полуоси орбиты фотоцентра от отношения массы одной из компонент, если массу другой принять постоянной (для верхнего рисунка фиксирована масса $0.35\,M_\odot$, для нижнего "--- $0.73\,M_\odot$).}
\end{figure}

Представим себе двойную систему с большой полуосью в 10 а.е., удаленную от Солнца на 25~пк. Предположим, что плоскость орбиты совпадает с картинной плоскостью. Используя PARSEC-изохроны для солнечной металличности, несложно вычислить величину большой полуоси орбиты фотоцентра:
\begin{equation}
\label{eq:Photocenter}
 a_{ph}=a\left| \frac{M_2}{M_1+M_2} - \frac{L_2}{L_1+L_2} \right|,
\end{equation}
где $M_1,M_2,L_1,L_2$ соответственно массы и светимости компонент.

Далее, варьируя массы компонент несложно проследить, в как меняется $a_{ph}$. Рисунок~\ref{fig:semiaxis1} демонстрирует два примера поведения $a_{ph}$. Верхняя панель данного рисунка фиксирует $M_2 = 0.35\,M_\odot$, а $M_1$ меняется от $0.1\,M_\odot$ до $0.8\,M_\odot$, нижняя "--- соответствует неизменной массе $M_2 = 0.73\,M_\odot$.

Как следует из рисунка~\ref{fig:semiaxis1}, нелинейность зависимости <<масса -- светимость>> ведет к ненулевым значениям $a_{ph}$ на всем интервале, кроме случая равенства масс компонент. Для многих значений $M_1$ величина большой полуоси орбиты фотоцентра превышает 40~mas. Эта величина вполне может быть обнаружена с помощью наземных ПЗС-наблюдений. Например, если орбита круговая, а разность эпох составляет половину орбитального периода (для масс  $0.35\,M_\odot$ и $0.73\,M_\odot$ это около 30 лет), $\Delta\mu\approx 13~mas/yr$, что вполне <<детектируемо>> при точности собственных движений около $4~mas/yr$.  

Но этот вывод не дает полного представления об эффективности поиска  $\Delta\mu$-двойных. Важно оценить общее количество двойных систем, которое можно выявить таким способом. В следующем разделе мы представим соответствующий анализ.

\subsection{Оценка количества $\Delta\mu$-двойных среди близких карликов}\label{subsec:ch1/sect2/sub2}
Задача оценки количества двойных систем, состоящих из маломассивных звезд и проявляющих себя как $\Delta\mu$-двойные, относится к числу естественных вопросов, которые возникают при построении наблюдательной программы. Необходимо отметить, что максимально несмещенные оценки возможны только в том случае, когда известны  точно статистические параметры звездного населения солнечной окрестности: пространственная плотность распределения, функция  масс, доля двойных систем в зависимости от массы и металличности, соответствующие распределения для отношения масс компонент и орбитальных параметров. На сегодняшний день параметры этих функций и распределений известны ненадежно. И тем не менее можно попытаться провести оценку, привлекая данные из научной периодики. Для построения модельной популяции маломассивных двойных систем мы использовали функцию масс из работы \cite{2005ASSL..327...41C},
распределение больших полуосей из статьи \cite{2010ApJS..190....1R}. Мы исходили из подсчетов звезд в солнечной окрестности, произведенных коллаборацией RECONS \cite{2019AJ....157..216W} и данных Gaia DR2. В итоге в области, ограниченной радиусом 50 пк, с учетом доли двойных для маломассивных звезд оказалось чуть больше 14 тысяч двойных систем с $M<0.7\,M_\odot$. 

В дальнейшем были образованы звездные пары, расстояния до которых, галактические широты и долготы назначались случайным образом так, чтобы модельная популяция заполняла весь объем внутри сферы радиусом 50~пк. Для каждой двойной системы назначались орбитальные параметры согласно упомянутым литературным источникам. С учетом масс и светимостей компонент вычислялись большие полуоси орбит фотоцентров. Далее образовывались массивы координат, моделирующие ряды наблюдений для каждой звезды с учетом характерных значений случайных ошибок отдельных положений звезд. Эпохи модельных наблюдений соответствовали реальным. В итоге несложно было вычислить количество звезд, которые можно было бы детектировать как $\Delta\mu$-двойные.  На рисунке~\ref{fig:F-hist} показана часть распределения величин $F$, демонстрирующая относительную редкость $\Delta\mu$-двойных. Подсчет дает около 1000 таких систем на всем небе. Учитывая то, что в пулковскую программу были включены все звезды в зоне склонений	$30^\circ$ -- $70^\circ$, мы могли рассчитывать на детектирование примерно 250 звезд. 

\begin{figure}[pt]\label{fig:F-hist}
\centering
\includegraphics[width=0.9\textwidth]{F-hist.png}
\caption{Распределение величины $F$, полученное в ходе численного моделирования. Пороговое значение $F=2.49$ показано вертикальной линией. }
\end{figure}

Проведенные оценки показывают, что обнаружение $\Delta\mu$-двойных звезд вполне реально на основе доступных данных. Относительно невысокие точности наземных наблюдений сильно  ограничивают такой способ выявления двойных систем. Стоит ожидать серьезного прогресса в этой области исследований по мере публикации релизов миссии Gaia.

Полезным будет сопоставление реально наблюдаемых  $\Delta\mu$-двойных с модельными оценками, проведенными в этом разделе, так как это даст дополнительную информацию для анализа статистических свойств ансамбля двойных систем, образованных маломассивными карликами.

\subsection{Потенциальное количество звезд, двойственность которых проявляется в деформированности их изображений на ПЗС-кадрах}\label{subsec:ch1/sect2/sub3}

В данной работе мы полагаемся не только на анализ собственных движений, но и принимаем во внимание возможность анализа формы изображений.  Модельные эксперименты с изображениями звезд показали, что успех детектирования сильно зависит от отношения звездных величин компонент, углового разделения и характерного размера изображения одиночной звезды, который можно оценить с помощью FWHM звездных изображений. Очевидно, что фотоцентры компонент интересующих нас пар звезд должны лежать внутри области ПЗС-кадра, задаваемой FWHM. Причем уверенное детектирование возможно при угловом разделении $\rho>0.5FWHM$.  Данная величина в используемом материале меняется от $\approx\,1"$ до $\approx\,3"$.  В качестве оценки сверху для числа звезд, двойственность которых можно обнаружить таким способом, можно принять количество объектов из модельной популяции, описанной в разделе~\ref{subsec:ch1/sect2/sub2}, с разделениями в пределах от $0.5FWHM<\rho<FWHM$. На рис.~\ref{fig:rho-hist} показана гистограмма распределения двойных систем по величине углового разделения компонент $\rho$. Вертикальные линии определяют указанные выше значения FWHM, характерные для используемых ПЗС-кадров. В итоге суммирование дает оценку "--- около 3000 звезд по всему небу.  

\begin{figure}[pt]\label{fig:rho-hist}
\centering
\includegraphics[width=0.9\textwidth]{rho-hist.png}
\caption{ Распределение угловых расстояний между компонентами для модельной популяции маломассивных двойных систем.}
\end{figure}

Однако следует помнить, что это всего лишь грубое приближение, которое должно совпасть с реальным значением по порядку величины. Полученный результат свидетельствует, что признаки двойственности могут быть обнаружены при анализе изображений примерно 20\,\% популяции двойных звезд.

\section{Анализ собственных движений, определенных на разных временных интервалах. Метод Вилена} \label{sec:ch1/sec3}
Как было показано в предыдущем разделе, выявление нелинейности движения по небесной сфере требует хорошей наблюдательной истории "--- нескольких десятков положений, полученных в разные эпохи. Свойства движений компонент двойных систем определили возможность массового поиска неразрешенных звездных пар без необходимости иметь большое количество точных положений. Достаточно иметь оценки собственных движений, полученные на разных временных интервалах. При высокой точности определения координат может хватать всего трех положений.

 Собственные движения определены для огромного количества звезд в ходе реализации разнообразных наблюдательных проектов. Это сделало возможным относительно массовое обнаружение звезд, которые имеют явные признаки двойственности. Высокая точность определения положений звезд с помощью  астрометрического спутника (порядка 1 mas) раскрыла новые пути поиска и исследования двойных систем \cite{1997ESASP1200.....E}. В 1999 году был представлен метод поиска неразрешаемых двойных систем, основанный на статистическом анализе наблюдаемых изменений собственных движений звезд \cite{1999A&A...346..675W}. В работе исследовались ярчайшие звезды, изученные в ходе миссии Hipparcos. Основная идея метода Вилена проиллюстрирована на рисунке~\ref{fig:widea}. Для физической одиночной звезды собственное движение, измеренное в течение короткого промежутка времени, в пределах точности измерений должно совпадать с собственными движением, полученным из очень длинного временного интервала. И в общем случае такого совпадения не будет наблюдаться для неразрешаемой двойной звезды. Из-за гравитационного влияния более слабой компоненты, движение фотоцентра двойной звезды будет волнообразным, что может определить значительное отличие мгновенно измеренного собственного движения от долгосрочного (в идеале "--- движение барицентра системы). Такую разницу в работе назвали <<космической ошибкой>> и обозначили как $\Delta\mu$. При значительном преобладании космической ошибки по сравнению с ошибкой измерения объект относился к кандидатам в двойные и обозначался как <<Дельта--мю двойная>> (<<$\Delta\mu$--binaries>>). Объекты, у которых космическая ошибка была в рамках ошибки наблюдений, были обозначены как <<кандидаты в одиночные звезды>>  (<<single--star candidate>>), если не было другой информации об их двойственной природе.

\begin{figure}[pt]
 \centering
 \includegraphics [scale=0.5] {Wielen-idea}
 \caption{Колебания фотоцентра двойной системы, вызванные влиянием орбитального движения, приводит к заметной разнице $\Delta\mu_{FH}$ между мгновенно измеренным собственным движением Hipparcos $\mu_{H}$ и средним собственным движением $\mu_{F}$ фотоцентра. Здесь период обращения двойной системы имеет среднюю длину ($\approx$\,30 лет), так что собственное движение $\mu_{F}$, полученное из наземных данных (например, из FK5), по существу равно собственному движению барицентра (cms) двойной звезды. Взято из \cite{1999A&A...346..675W}, Рис. 1.}
 \label{fig:widea}
\end{figure}

В качестве квази-мгновенных были взяты высокоточные собственные движения HIPPARCOS \cite{1997ESASP1200.....E}, $\mu_{H}$, полученные за период около 3 лет в 1991 году. В качестве квази-средних "--- собственные движения, полученные несколькими способами. 

В первую очередь для вычисления квазисредних (<<долгосрочных>>) собственных движений использовался каталог FK5 \cite{1988VeARI..32....1F}, \cite{1991VeARI..33....1F}. Из него были извлечены собственные движения ($\mu_F$), а также положения звезд для вычисления новых собственных движений совместно с положениями HIPPARCOS ($\mu_{0F}$), временная база которых составила в среднем 40 лет. Ошибки вычисленных собственных движений оказались значительно меньше, чем указанные в FK5. Их точность обеспечили относительно маленькие ошибки положений FK5 и значительная разница эпох FK5 и HIPPARCOS. Таким образом были получены по 3 разности собственных движений для каждой из координат $\delta$  и $\alpha ^*$~=~$\alpha\,\cos\delta$. Стоит отметить, что положения, взятые из FK5, для одной звезды могли не совпадать по эпохе для разных координат. Это несколько усложнило физическую интерпретацию результатов исследования, однако исключило корреляции в собственных движениях по разным координатам.

Помимо каталога FK5 в исследовании был использован каталог GC \cite{1936gcts.book.....B}. Хотя количество объектов GC (33\,342) сильно больше, чем в FK5 (4\,652), низкая точность собственных движений этого каталога не позволила использовать собственные движения, опубликованные в GC. Однако огромная разница эпох наблюдения с HIPPARCOS позволила получить новые собственные движения ($\mu_{0}$(GC)). Также стоит отметить, что большое пересечение выборки FK5 и GC обеспечило проверку согласованности данных.

Для вычисления ошибок получаемых $\Delta\mu$ авторы использовали ошибки собственных движений из каталога HIPPARCOS, а также комбинацию индивидуальных ошибок собственных движений наземных каталогов со значениями систематических ошибок редукции каталогов в систему HIPPARCOS.

В ходе исследования статистических особенностей данных авторы учли тот факт, что собственные движения HIPPARCOS по двум координатам коррелируют. По данным в каталоге коэффициентам корреляции была построена ковариация, которая позволила скорректировать (повернуть на угол $\psi$) оси собственных движений таким образом, чтобы они соответствовали осям эллипсоида ошибок (см. рисунок~\ref{fig:werr}). Вдоль новых осей авторами были определены формулы дисперсий ожидаемого гауссовского распределения ошибок.

Для оценки статистической значимости $\Delta\mu$ был выведен тестовый  параметр оценки $F_{0H}$:

\begin{equation}
  \label{eq:WiF}
  F^{2}_{0H} =\left(\frac{\Delta\mu_{0H,\psi}}{\epsilon_{\Delta\mu_{0H,\psi}}}\right)^{2}+\left(\frac{\Delta\mu_{0H,\bar{\psi}}}{\epsilon_{\Delta\mu_{0H,\bar{\psi}}}}\right)^{2}.
\end{equation}

Если звезда не является двойной, то ожидается, что некоррелированные переменные $\Delta\mu_{0H,\psi}$ и  $\Delta\mu_{0H,\bar{\psi}}$ будут следовать нормальным распределениям со средним нулем и дисперсиями, определенными авторами ранее. В этом случае вероятность $W(F)$ случайно найти значение $F_{0H}$, равное или превышающее наблюдаемое значение, определялось уравнением:

\begin{equation}
  \label{eq:WiW}
  W(F) = e^{-F^2_{0H}/2}
\end{equation}

Плотность вероятности $w(F)dF$, показывающая вероятность нахождения $F_{0H}$ между $F$ и $F+dF$ определяется как:

\begin{equation}
  \label{eq:Wiww}
  w(F) = -\frac{dW(F)}{dF} = F_{0H}\,e^{-F^2_{0H}/2}
\end{equation}

Характер поведения этих функций (см. рисунок~\ref{fig:wWw}) дал основание полагать, что высокое наблюдаемое значение $F_{0H}$ является значимым показателем двойственной природы исследуемого объекта. В качестве минимального значения $F_{0H}$, при котором звезду относили к $\Delta\mu$-двойным, авторы определили как $F_{lim,b}=3.44$, который дает $W(3.44)=0.0027$, что удовлетворяет критерию 3$\sigma$. Вторым пограничным значением $F_0H$, ниже которого звезда считалась кандидатом в одиночную стало $F_{lim,s}=2.49$, что соответствует критерию 2$\sigma$ ($W(2.49)=0.0456$).

 \begin{figure}[pt]
 \centering
 \includegraphics [scale=0.5] {Wielen-err}
 \caption{Эллипсоид ошибок измерений $\Delta\mu$, наклонен относительно экваториальной системы ($\delta$, $\alpha^*$) на угол $\psi$. Большая ось эллипсоида ошибок указывает в направлении $\psi$, малая ось "--- в направлении $\bar{\psi}$.  Взято из \cite{1999A&A...346..675W}, Рис. 2.}
 \label{fig:werr}
\end{figure}

Тогда как для звезд представленных только в CG есть только одно значение $F_{0}(GC)$, для звезд FK5 в общем случае доступно 3 параметра оценки ($F_{FH}$, $F_{0H}$, $F_{0F}$). Авторы предлагают считать объект $\Delta\mu$-двойным, если хотя бы 1 из величин больше, чем $F_{lim,b}$. Для кандидата в одиночную звезду, все значения должны быть меньше $F_{lim,s}$.

Авторы отмечают, что данный метод имеет ограничения. К примеру, он не применим к двойным звездам, чей период орбитального движения менее 3 лет, а для звезд с очень большими периодами (порядка 1000 лет) метод требует чрезвычайно высокой точности. Однако, он хорошо работает для близких быстрых звезд, чей период составляет десятки лет.  В результате проведенной работы было обнаружено больше тысячи впервые детектированных $\Delta\mu$-двойных звезд, что составило примерно 10\,\% от исследованных.

 \begin{figure}[pt]
 \centering
 \includegraphics [scale=0.5] {Wielen-Ww}
 \caption{Функция W(F) описывает вероятность случайного нахождения наблюдаемого значения тестового параметра, большего чем F. Функция w(F) "--- плотность вероятности. Указаны два критических значения: F~>~3.44 для $\Delta\mu$-двойных и F~<~2.49 для кандидатов в одиночную звезду. Взято из \cite{1999A&A...346..675W}, Рис. 3.}
 \label{fig:wWw}
\end{figure}

Напомним, что исследования Вилена и его коллег затрагивают яркие звезды из состава FK5 и HIPPARCOS. В основном это объекты в диапазоне от солнцеподобных звезд до звезд ранних спектральных классов главной последовательности и гигантов, распределенные в ближайших 100 пк от Солнца. Объекты нашего исследования "--- звезды-карлики ближайшего окружения Солнца. Но, с некоторыми усовершенствованиями, идея Вилена вполне применима к поиску двойных систем для этого типа объектов. 

\section{Исследования близких карликов в Пулковской обсерватории} \label{sec:ch1/sec4}
В последнее десятилетие в Пулковской обсерватории активно реализуется комплексная программа изучения звезд с большими собственными движениями, включающая определение тригонометрических параллаксов  \cite{2010AstL...36..576K}, \cite{2013MNRAS.435.1083K}, уточнение собственных движений, анализ кинематики. В том числе реализуется изложенный выше подход к детектированию двойных карликов. Метод Вилена был адаптирован для исследования звезд низкой светимости с большими значениями собственного движения. Наиболее подробное изложение адаптации и реализации метода Вилена в пулковской программе будет представлено в главе~\ref{ch:ch3}. Первая попытка применения данного подхода \cite{2011AstL...37..420K} была предпринята на материале звезд-карликов, расположенных в зенитной зоне Пулковской обсерватории (от $30^\circ$ до $70^\circ$ по склонению). Наблюдения были проведены с помощью Нормального астрографа. В наблюдательную программу вошли 1123 звезды с большими собственными движениями ($\mu$~>~300~mas/yr). Следующая работа \cite{2015AstL...41..833K} содержит исследования практически всех быстрых звезд, относимых к категории близких карликов и доступных для наблюдений в Пулкове. Для дальнейшего анализа были вычислены собственные движения, однако в отличие от первой реализации положения звезд брались не из каталогов, а были получены методами прямой редукции с кадра на кадр, это позволило избежать систематических ошибок каталогов, однако встал вопрос о реализации способа вычисления пиксельных координат. На первом этапе использовался нелинейный МНК, однако в дальнейшем для определения параметров PSF хорошо зарекомендовал себя адаптированный для исследования изображений звезд метод shapelet-формализма, описанию которого посвящена следующая глава. Этот метод дал возможность осуществить выявление двойных звезд на основе особенностей их изображений, о чем подробнее сказано в главе ~\ref{ch:ch4}.