%% Согласно ГОСТ Р 7.0.11-2011:
%% 5.3.3 В заключении диссертации излагают итоги выполненного исследования, рекомендации, перспективы дальнейшей разработки темы.
%% 9.2.3 В заключении автореферата диссертации излагают итоги данного исследования, рекомендации и перспективы дальнейшей разработки темы.
\begin{enumerate}
 \item Для выполнения поставленных задач были проведены наблюдения на Нормальном астрографе и телескопе <<Сатурн>>;
 \item Для обработки полученных наблюдений был исследован и адаптирован метод shapelet-разложения;
   \item Для получения прочих материалов исследования, численных расчетов и построения промежуточных моделей было создано специализированное программное обеспечения на C++ и Python;
 \item На основе анализа собственных движений 1308 быстрых звезд было выявлен 121 кандидат в $\Delta\mu$-двойные, по данному исследованию опубликована работа \cite{2015AstL...41..833K};
 \item Были проведены дополнительные спекл-интерферометрические исследования 7 программных звез, для пяти из которых были подтверждены статусы двойных (присутствие 2х из них в каталоге WDS может служить фактором верификации исследования), по выявлению одной из звезд опубликована работа \cite{2016AstL...42..686K} и уже получены наблюдения, позволяющие построить её предварительную орбиту;
   \item В результате анализа форм изображений 702 звезд (отмеченных в Gaia DR2 флагом \glqq duplicate source\grqq ) было выявлено ещё 138 кандидатов в двойные, по данному исследованию также написана статья \cite{2018AstL...44..103K}.
\end{enumerate}
